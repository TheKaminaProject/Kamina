 
\documentclass[12pt]{article}
\usepackage[utf8]{inputenc}
\usepackage[english]{babel}
\usepackage{hyperref}
\usepackage[a4paper, total={6in, 8in}]{geometry}


\hypersetup{
    colorlinks=true,
    linkcolor=blue,
    filecolor=magenta,      
    urlcolor=blue,
}
 
\urlstyle{same}
\usepackage{graphicx}


\usepackage{array}
\newenvironment{conditions}
  {\par\vspace{\abovedisplayskip}\noindent\begin{tabular}{>{$}l<{$} @{${}={}$} l}}
  {\end{tabular}\par\vspace{\belowdisplayskip}}
\begin{document}

\title
{
    Kamina \\
    REST API Specifications \\
}
\author
{
    The Kamina Project \\
    \href{https://github.com/TheKaminaProject/Kamina}{https://github.com/TheKaminaProject/Kamina} \\ 
}
\maketitle



\filbreak
\section{Global meta-data}
\begin{verbatim}

curl -X GET http://localhost:9001/api/graph

    {
        "root": "ID"
    }

\end{verbatim}

The server name, description and other metadata is not needed to be saved elsewhere, it would be redundant, that same data is actually the data of the Root Room.

Return Codes:
\begin{verbatim}
    200 if Root Room exists
    404 if Root Room does not exist
    403 if Root Room does not have public permissions
\end{verbatim}


\filbreak
\section{Rooms}

\subsection{Get Room Info}
\begin{verbatim}

curl -X GET http://localhost:9001/api/graph/ID

    {
        "name": "string"
        "description: "string"
        //TODO
    }



\end{verbatim}
Return Codes:
\begin{verbatim}
    200 if Room exists and user is allowed
    404 if Room does not exist
    403 if Room does not have public permissions
\end{verbatim}




\subsection{Change Room name}

\begin{verbatim}

curl -H "Content-Type: application/json" -X POST http://localhost:9001/api/graph/ID -d 

    { 
        "name": "newnamestring" 
    }

\end{verbatim}

Return Codes:
\begin{verbatim}
    200 if Room exists and name is correct
    404 if Room does not exist
    403 if forbidden to change name
\end{verbatim}


// create room returns Room ID


\section{Messages}



\end{document}
