 
\documentclass[12pt]{article}
\usepackage[utf8]{inputenc}
\usepackage[english]{babel}
\usepackage{hyperref}
\usepackage[a4paper, total={6in, 8in}]{geometry}


\hypersetup{
    colorlinks=true,
    linkcolor=blue,
    filecolor=magenta,      
    urlcolor=blue,
}
 
\urlstyle{same}
\usepackage{graphicx}


\usepackage{array}
\newenvironment{conditions}
  {\par\vspace{\abovedisplayskip}\noindent\begin{tabular}{>{$}l<{$} @{${}={}$} l}}
  {\end{tabular}\par\vspace{\belowdisplayskip}}

\newcommand{\bigO}{\mathcal{O}}
\begin{document}

\title
{
    Kamina \\
    REST API Specifications \\
}
\author
{
    Federico Santamorena \\
    \href{https://github.com/yatima1460/Kamina}{https://github.com/yatima1460/Kamina} \\ 
}
\maketitle



\section{Global meta-data}

\begin{verbatim}

curl -X GET http://localhost:9001/api/graph

    {
        "root": "id"
    }

\end{verbatim}

The server name, description and other metadata is not needed to be saved elsewhere, it would be redundant, that same data is actually the data of the Root  Room.

Return Codes:
\begin{verbatim}
    200 if Root Room exists
    404 if Root Room does not exist
    403 if Root Room does not have public permissions

\end{verbatim}




\section{Rooms}


\begin{itemize}
    \item POST /api/graph/create\_room
    \begin{itemize}
        \item returns Room ID
    \end{itemize}
    \item POST /api/graph/ID/name
    \begin{itemize}
        \item change Room name
    \end{itemize}
    \item GET /api/graph/ID/name
    \begin{itemize}
        \item get Room name
    \end{itemize}
\end{itemize}

\begin{itemize}
    \item POST /api/room/ID
    \begin{itemize}
        \item Owner
    \end{itemize}
\end{itemize}

\begin{itemize}
    \item GET /api/room/ID
    \begin{itemize}
        \item Owner
    \end{itemize}
\end{itemize}



\section{Messages}



\end{document}
